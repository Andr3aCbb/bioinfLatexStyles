\documentclass[12pt,titlepage]{article}

%% THE USEPACKAGES NECESSARY FOR THIS EXAMPLE
%% NOTE THAT genetics_manu_style MUST BE CALLED AFTER mychicago
\usepackage{graphicx}
\usepackage{endfloat}
\usepackage{amsfonts}
\usepackage{mychicago}
\usepackage{subfigure}
\usepackage{genetics_manu_style}



%% THE MANUSCRIPT TITLE
\title{Improving the Estimation of Bacterial Allele Frequencies  \\
\protect\small A Letter Submitted to {\em Genetics}}





%%  THE AUTHOR DECLARATIONS.  USE \THANKS TO GIVE ADDRESSES
\author{Eric C. Anderson,\thanks{Interdisciplinary Program in Quantitative
Ecology and Resource Management, University of Washington,
Seattle, WA 98195} Paul A. Scheet\thanks{Department of Statistics, University of
Washington, Seattle, WA 98195}}



%% USE \RENEWCOMMAND FOR CORRESPONDING ADDRESS, RUNNING HEAD
%% AND KEYWORDS, ETC.
\renewcommand{\CorrespondingAddress}{Department of Statistics \\
    University of Washington \\ 
    Box 354322\\ 
    Seattle, WA 98195 \\ 
    (206) 685-8969 (ph.)  \\
    (206) 685-7419 (fax)  \\ 
    \texttt{eriq@stat.washington.edu} \vfill}
\renewcommand{\RunningHead}{Bacterial Allele Frequencies}
\renewcommand{\CorrespondingAuthor}{Eric C. Anderson}
\renewcommand{\KeyWords}{Presence/Absence Data, Constrained Optimization}



%% SOME COMMANDS FOR THE CONTENT OF THIS FILE.  NOT NECESSARY FOR
%% GENETIC_MANU_STYLE
\newcommand{\bp}{\mathbf{p}}
\newcommand{\LLL}{\mathcal{L}}




%% BEGIN DOC
\begin{document}


\maketitle



%% ABSTRACT ENVIRONMENT
\begin{abstract}
\citeN{Rannalaetal2000} develop a useful Poisson process
model for estimating allele frequencies in bacterial populations using
presence/absence data on infected hosts.  However, they also show that the
estimators they derive suffer from bias.  In this letter we derive
different estimators from the same model and show these estimators to
be preferable.  This paper right now is intended merely as an illustration
of how to use the package "genetics\_manu\_style.sty."
\end{abstract}


\citeN{Rannalaetal2000} develop a useful Poisson process
model for estimating allele frequencies in bacterial populations using
presence/absence data on infected hosts.  However, they also show that the
estimators they derive suffer from bias.  In this letter we derive
different estimators from the same model and show these estimators to
be preferable.

\citeN{Rannalaetal2000}   write the log likelihood function for allele
frequencies 
$\bp=(p_1,\ldots,p_k)$ and infection rate
$\lambda$ as 
\begin{equation}
\ell = -n \log(1-e^{-\lambda}) + \sum_{j=1}^k
\{z_j \log(1-e^{-\lambda p_j}) - (n-z_j)\lambda p_j\}
\label{eq:loglike}
\end{equation}
where $z_j$ is the number of sampled hosts infected by
bacteria with allele $j$, $j=1,\ldots,k$, and where $n$ is the total
number of {\em infected} hosts sampled.  Then, although they note that a
more rigorous statistical approach would be to find estimators by
maximizing the likelihood jointly over the parameters $\bp$ and
$\lambda$, they employ an heuristic maximization method to derive their
estimators.  Their method does not obtain the maximum likelihood
estimators for $\bp$ and $\lambda$, because it does not properly account
for the constraint that the allele frequencies sum to unity. 


The true maximum likelihood estimators (MLE's) may be found by performing
the constrained optimization by the method of LaGrange multipliers.  We
show this below.  The MLE's derived this way are shown to be
far less biased than the estimators given in \citeN{Rannalaetal2000}. Additionally, we
show that if data are available on both infected and uninfected hosts, then it is
preferable to use all that information to estimate $\bp$ and
$\lambda$, rather than using only the data on infected hosts. 

\section{Constrained Optimization}
Any values of $\bp$ and $\lambda$ which maximize
(\ref{eq:loglike}) while satisfying the constraint $\sum_{j=1}^k p_j =1$,
will also be maximizers of the equation
\begin{equation}
\LLL = -n \log(1-e^{-\lambda}) + \sum_{j=1}^k
\{z_j \log(1-e^{-\lambda p_j}) - (n-z_j)\lambda p_j\} + \mu
(\sum_{j=1}^k p_j - 1)
\label{eq:LaGrange}
\end{equation}
where $\mu$ is an arbitary non-zero constant ($\mu$ is called the
``LaGrange multiplier" in this context).  The MLE's may be found
by setting the first partial derivatives of (\ref{eq:LaGrange}) with
respect to $\bp$ and
$\lambda$ equal to zero and solving for $\bp$, $\lambda$ and $\mu$.  Thus
we have
\begin{eqnarray}
0 = \frac{\partial\LLL}{\partial p_j} & = &\frac{z_j \lambda e^{-\lambda
p_j} }{1-e^{-\lambda p_j}} - (n - z_j)\lambda + \mu~~~~~~~,~~~~~~j =
1,\ldots, k \label{eq:dbyp}\\
0 = \frac{\partial\LLL}{\partial \lambda} & = & 
\frac{-ne^{-\lambda} }{1-e^{-\lambda}} + \sum_{j=1}^k  p_j
\biggl\{\frac{z_j  e^{-\lambda
p_j} }{1-e^{-\lambda p_j}} - (n - z_j) \biggr\}. \label{eq:dbylambda}
\end{eqnarray}
By adding and subtracting terms of
$(p_j\mu)/ \lambda$, (\ref{eq:dbylambda}) may be rewritten as 
\begin{equation}
0 = \frac{\partial\LLL}{\partial \lambda}  =  
\frac{-ne^{-\lambda} }{1-e^{-\lambda}} + \sum_{j=1}^k  p_j
\biggl\{\frac{z_j  e^{-\lambda
p_j} }{1-e^{-\lambda p_j}} - (n - z_j) + \frac{\mu}{\lambda} \biggr\} -
\sum_{j=1}^k \frac{p_j \mu}{\lambda}.
\label{eq:subtle}
\end{equation}
In (\ref{eq:subtle}), the part in curly braces is equal to
$\frac{1}{\lambda}\frac{\partial\LLL}{\partial p_j}$, which is
equal to zero.  So, we may solve for $\mu$:
\begin{eqnarray}
0 & =  & 
\frac{-ne^{-\lambda} }{1-e^{-\lambda}}  -
 \frac{ \mu}{\lambda} \nonumber  \\
\mu & = & \frac{-\lambda n e^{-\lambda}}{1-e^{-\lambda}}. \nonumber
\end{eqnarray}
By substituting this value for $\mu$ into (\ref{eq:dbyp}) the MLE for
$p_j$ is found to be 
\begin{equation}
\hat{p}_j = -\frac{1}{\lambda} \log\biggl( \frac{ n/(1-e^{-\lambda}) -
z_j}{n/(1-e^{-\lambda})}\biggr)~~~~~~,~~~~~~j = 1,\ldots,k
\label{eq:phat}
\end{equation}
and the MLE for $\lambda$ follows from the fact that the $p_j$ sum to
one
\begin{equation}
\hat{\lambda} = -\sum_{j=1}^k \log \biggl( \frac{ n/(1-e^{-\lambda}) -
z_j}{n/(1-e^{-\lambda})}\biggr).
\label{eq:lambdahat}
\end{equation}
The system of equations (\ref{eq:phat}) and (\ref{eq:lambdahat}) cannot
be solved explicitly, but it is straightforward to find $\hat\bp$
and $\hat{\lambda}$ iteratively to arbitrary precision.  For reasons that will
become clear shortly, we call these
estimators the {\em partial sample MLE's}.

When $\lambda$ is large, so that most of the sampled hosts are infected,
this MLE will differ little from the estimator derived by
\citeN{Rannalaetal2000}, since $1-e^{-\lambda}$ will be very close to one
in that case. It will differ more from the estimator in
\citeN{Rannalaetal2000} when $\lambda$ is small  and many of the hosts
sampled are not infected by any bacteria.  
We have empirically found that this MLE usually gives estimates that are
between the estimates from \citeANP{Rannalaetal2000}'s estimator and the ``uncorrected
estimates"  of $z_j/\sum_{i=1}^k z_i$.  The MLE does not fall between the other
two estimators only at alleles for which all three estimators give 
very similar frequency estimates.
Hence, while the uncorrected estimator used in \citeN{Wangetal1999} and
\citeN{Qiuetal1997} tends to overestimate low-frequency alleles and underestimate
high-frequency alleles, the estimator proposed by \citeN{Rannalaetal2000} over-corrects
the problem, underestimating low-frequency alleles and overestimating high frequency
alleles.  As our simulations show, the estimators we propose don't suffer from this
bias.  

Before we jump to the next section, we will demonstrate several different heading levels:

\Genetics2level{Genetics 2nd-Level Section Heading}
This section level heading is apparently seldom used by anyone or anything in {\em
Genetics} (according to their own account on their website.  You invoke it with the
command $\backslash$\texttt{Genetics2level}.  It is better to use the other two
subheadings:

\subsection{The Subsection Heading:}
This pleasing little subheading is obtained with a $\backslash$\texttt{subsection}
command.  It is most commonly used as a second-level heading beneath the centered,
small-caps heading.  You have to add your own colon!!

\subsubsection{The lowest-level subheading:}  This is the lowest-level subheading which
is obtained with the $\backslash$\texttt{subsubsection} command.  Once again, you have
to add your own colon.  

\section{Estimation from the whole sample of hosts}
The quantity $n/(1-e^{-\lambda})$ can be seen to estimate the total number
of hosts sampled, both infected and uninfected.  In fact, it is possible
to improve the estimation of $\bp$ and
$\lambda$ by including information (if available) on the number
of uninfected, as well as infected hosts sampled.  
Since \citeN{Rannalaetal2000} use only the infected hosts, their log-likelihood
(\ref{eq:loglike}) contains the term $-n\log(1-e^{-\lambda})$ which arises from
conditioning upon the hosts being infected.  If one uses the information in both
infected and uninfected hosts, that term vanishes from the log-likelihood.  
 Doing so, and letting
$M$ denote the total number of hosts (infected and uninfected) sampled, the
log-likelihood becomes
\[
\ell = \sum_{j=1}^k
\{z_j \log(1-e^{-\lambda p_j}) - (M-z_j)\lambda p_j\}.
\]
Since the $p_j$ and $\lambda$ only occur together as the products
$\lambda p_j$ in this expression, the unconstrained maximization
procedure pursued by \citeANP{Rannalaetal2000} would be appropriate, and it
follows that the MLE's (which we will call the {\em complete sample MLE's}) are
\begin{eqnarray}
\hat{p}_j & = &  -\frac{1}{\lambda} \log\biggl( \frac{ M
- z_j}{M}\biggr)~~~~~~,~~~~~~j = 1,\ldots,k
\label{eq:phatU}   
 \\
\hat{\lambda} & = & -\sum_{j=1}^k \log \biggl( \frac{ M -
z_j}{M}\biggr).
\label{eq:lambdahatU}
\end{eqnarray}
These expressions are identical to those in \citeN{Rannalaetal2000} except that $M$ is
used in place of $n$.

 If $M$, the total number of hosts sampled, is available, these are
the preferred estimators for two reasons.  First, being available in closed
form, it is not necessary to iteratively compute the complete sample MLE's   as it is for
the partial sample MLE's.  And second, it can be shown that if $M$ is known, and
$\lambda$ unkown, then $M$ and the set of $z_j$ ($j=1,\ldots,k$) are the minimal
sufficient statistics for the parameters
$\lambda$ and $\bp$. Since $M$ is not a deterministic function of $n$ it follows from
the definition of minimal sufficiency that
$n$ and the set of $z_j$ $(j=1,\ldots,k)$ are not sufficient for $\lambda$
and $\bp$,  and so the partial sample MLE is not based on all the available information
unless the investigator failed to record or does not actually know $M$.  Since
estimators based on sufficient statistics typically have smaller variance than those
that are not based on sufficient statistics, the complete sample MLE should be used when
$M$ is known.  Of course, if only $n$ and the $z_j$'s are known, then, in that context,
the partial sample MLE {\em is} based on the sufficient statistic and should be used. 
Simulations (Figure~\ref{fig:VarGraph}) confirm that the complete sample MLE has lower
variance than the partial sample MLE.

Programs to compute the partial sample and complete sample MLE's described above may
be downloaded from \texttt{http://www.rannala.org}.  

\section{Simulations and Data}
We have repeated a small set of simulations like those carried out in
\citeN{Rannalaetal2000}.   Briefly, hosts were infected at rate $\lambda=2$ by $k=3$
different allelic types.  Hosts
were sampled from this population until $n=100$ infected hosts had been sampled.  The
total number,
$M$, of hosts sampled to achieve 100 infected ones was also recorded for each simulated
sample so that we could compute the complete sample MLE.  Simulations were done for nine
different sets of allele frequencies in which 
$p_1$ was taken from $\{.1,.2,\ldots,.9\}$ and $p_2=p_3=.5(1-p_1)$.  For each set of
allele frequencies, 50,000 replicate samples were drawn and the various estimators were
computed for each one.  Occasionally, when $p_1$ was large, a sample would be
drawn with $z_1 = n$.  In such a case the partial sample MLE is undefined, and
so we discarded these samples (thus, 9 were discarded when $p_1=.8$ and 1483 when
$p=.9$).  The results are summarized in Figures~\ref{fig:BiasVar}
and~\ref{fig:VarGraph}.  They demonstrate that the MLE's we have derived here are much
less biased than the estimators given in \citeN{Rannalaetal2000}, and they demonstrate
that the variance of the complete sample MLE is smaller than that of the partial sample
MLE (albeit only slightly for $\lambda=2$). 

We have also reanalyzed the tick data given in Table~1 of \citeN{Rannalaetal2000} using
the partial sample MLE.  Our estimates of the allele frequencies for all of the
alleles at {\em ospA} and all of the alleles except allele~D at {\em ospC} were
intermediate to the ``uncorrected estimate" and the \citeANP{Rannalaetal2000}
estimate.  The three different estimates for allele~D at {\em ospC} differed only at
the fourth decimal place.  Our method estimates that the {\em Borrelia burgdorferi}
allele frequencies are somewhat more uniform than inferred by \citeN{Rannalaetal2000}.

\begin{figure}
\begin{center}
\mbox{
\subfigure[Estimate of Lambda]{\includegraphics*[width = .4\textwidth]
	{LambdaGraph.eps}} \hspace*{.1\textwidth}
\subfigure[Standardized Bias]{\includegraphics*[width = .4\textwidth]
	{BiasGraph.eps}}
}
\end{center}
\caption{Figures from simulations like those in \protect\citeN{Rannalaetal2000} for the
case
$\lambda = 2$.  ``R" indicates the estimator from
\protect\citeANP{Rannalaetal2000}\  ``P" denotes the partial sample MLE
(Equations~\ref{eq:phat} and~\ref{eq:lambdahat}), and ``C" denotes the complete sample
MLE (Equations~\ref{eq:phatU} and~\ref{eq:lambdahatU}).  We show results for just one
sample size, $n=100$  infected hosts. For the ``C" estimates, the additional
uninfected hosts are used in the estimation.  (a) Mean values of the estimates for
$\lambda$ from data on a three allele locus with
$p_2 = p_3 = .5(1-p_1)$ based on 50,000 simulated replicates.  The ``R" estimator
consistently overestimates $\lambda$, while both the partial and complete sample
estimators estimate $\lambda$ well (though the partial sample MLE tends to overestimate
$\lambda$ when the allele frequencies are highly uneven).  (b) Percent standardized bias
for the estimate of
$p_1$ in the three allele locus described above. The arching curve is the same found in
Figure~1 of
\protect\citeN{Rannalaetal2000}.  The other two curves show that the estimators
we propose estimate
$\bp$ with very little bias.}
\label{fig:BiasVar}
\end{figure}

\begin{figure}
\begin{center}
\includegraphics*[width=.4\textwidth]{VarianceGraph.eps}
\end{center}
\caption{Estimated variance of the partial and complete sample MLE's based on 50,000
replicates.  The variance of the complete sample MLE is lower, especially for uneven
allele frequencies.}
\label{fig:VarGraph}
\end{figure}

\section{Acknowledgments}
We thank the members of the Statistical Genetics seminar group in the Departments of
Statistics and Biostatistics at the University of Washington for helpful discussions,
particularly Elizabeth Thompson, Matthew Stephens, and Anne-Louise Leutenegger.


\bibliography{genetics_example}
\bibliographystyle{mychicago}
\end{document}